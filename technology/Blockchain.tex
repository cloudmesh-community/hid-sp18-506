\section{Blockchain}

A blockchain is a continuously growing list of linked records, called blocks. The most recent transactions are recorded and added to it in chronological order, it allows the public to keep track of all transactions in the chain without central recordkeeping.  Each computer participating in the blockchain gets a copy of the data which is downloaded automatically. 

Blockchains are decentralized by design and provide a new level of the way we trust data. For use as a distributed ledger, a blockchain is typically managed by a peer-to-peer network on the internet collectively agreeing to a protocol for validating new blocks. Once recorded, the data in any given block cannot be changed. One computer trying to tamper a record would have to beat all the computers in the network that are verifying the transactions. Thus tampering is very close to impossibility.

Blockchains "are an example of a distributed computing system with high Byzantine fault tolerance. Decentralized consensus has therefore been achieved with a blockchain. This makes blockchains potentially suitable for the recording of events, medical records, and other records management activities, such as identity management, transaction processing, documenting provenance, food traceability or voting"  ~\cite{hid-sp18-506-Blockchain}.
